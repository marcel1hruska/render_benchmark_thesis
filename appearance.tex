\chapter{Appearance Computations}
\label{chap:appearance}

So far, we've covered the fundamental basics of the light and the rendering process to be able to comprehend the more advanced techniques practiced in the computer graphics. As we've mentioned before, our primary goal is to evaluate the computational accuracy of several specific appearance sensations. Even though they are quite common in every day life, their integration to the modern renderers is, to this date, rare.  In this chapter, we discuss these phenomena individually --- their manifestations in the nature, the physics behind them and finally, their computations in the rendering process. 

\section{Reflectance}

The reflective surfaces are a surprisingly common sighting. As the perfectly diffuse materials basically do not exist in the nature, a large set of the materials that surround us are considered glossy. In \autoref{sec:BRDF}, we explained the bidirectional reflectance distribution function that defines the reflective properties of a material. 

\subsection{Fresnel equations}

It is necessary to know the basics of the geometry optics to be able to properly define a reflectance model. First of all, the \emph{Snell's law}~\cite{pharr2016physically}

\begin{equation}
\eta_i sin\theta_i = \eta_t sin\theta_t 
\end{equation}

states that the incoming angle $\theta_i$ (angle between the surface normal and the incoming direction) times the \emph{index of refraction} of the entering medium $\eta_i$ must be equal to their transmitted counterparts. In other words, knowing the indices of refraction of the entering and the leaving media and the incoming direction, we can compute the transmitted direction.

The index of refraction (IOR) varies from material to material (e.g. IOR of glass is $\sim$1.5) and it essentially states the ratio between the speed of light in the vacuum and speed of light in the current medium: 

\begin{equation}
n=\frac{c}{v}
\end{equation}

, where $n$ is the IOR, $c$ is the speed of light in the vacuum and $v$ is the phase velocity of light in the current medium

However, this gives us only the direction of the refracted light. In most cases, we also need to know the ratio between the amount of reflected and refracted light. Depending on the polarization of the light (further explained in \autoref{sec:polarization}), the \emph{Fresnel equations}~\cite{pharr2016physically} take the two following forms:

\begin{align*}
r_s = \frac{\eta_t cos\theta_i - \eta_i cos\theta_t}{\eta_t cos\theta_i + \eta_i cos\theta_t}\\
r_p = \frac{\eta_i cos\theta_i - \eta_t cos\theta_t}{\eta_i cos\theta_i + \eta_t cos\theta_t} 
\end{align*}

From these, we can compute the \emph{Fresnel reflectance} for an unpolarized light:
\begin{equation}
F_r=\frac{1}{2}(r_s^2 + r_p^2)
\end{equation}

The transmitted energy is equal to $1-F_R$ due to the energy conservation law.

Note that the previous computations describe only a subset of the materials called the \emph{dielectrics}, which are the materials that do not conduct electricity and are capable of transmitting light, such as glass, water, diamond, etc. The second large group consists of the \emph{conductors} which are basically all metals/materials with opaque surfaces. Actually, the light is also transmitted into the conductor, however, due to its physical properties, it is quickly absorbed. There exists a third group of the \emph{semiconductors} which are very rarely considered in the physically based rendering and therefore we skip them. A comparison between a dielectric and a conductor is shown in \autoref{fig:compare_dielectric_conductor}.

\begin{figure}[h]
	\centering
	\begin{tabular}{cc}
		\includegraphics[width=.4\linewidth]{img/dielectric_diamond.jpg}
		&
		\includegraphics[width=.4\linewidth]{img/conductor_aluminium.jpg}
	\end{tabular}
	\caption{A preview of a dielectric (diamond, left) a  conductor (aluminium, right) rendered in Mitsuba2~\cite{mitsubaWeb}}
	\label{fig:compare_dielectric_conductor}
\end{figure}

While this is a matter of simply computing the Fresnel equations, the practice is more complicated as most of the commonly seen materials are not perfectly smooth. They may be created rough on purpose or slight imperfections due to the manufacturing errors cause that the surfaces are generally rough.

\subsection{Microfacet theory}
With that in mind, the \emph{microfacet theory} was developed by \citet{cook1982reflectance} to address this aspect and provide a theoretical representation of the rough surfaces.

The main idea is that a rough surface consists of the \emph{microfacets} -- a collection of very small surfaces distributed statistically throughout the whole underlying \emph{macrosurface}. The aggregate behavior of the computed values for each of these microfacets determines the final scattering. An example of such distribution is shown in \autoref{fig:microfacets}.

\begin{figure}[h]
	\centering
	\includegraphics[width=.8\linewidth]{img/microfacets.pdf}
	\caption{A demonstration of a very rough (left) and a relatively smooth (right) microfacet distribution~\cite{pharr2016physically}}
	\label{fig:microfacets}
\end{figure}

As these microfacet computations are local, we need to consider the possibility that they might obscure each other. Three main aspects are accounted for:
\begin{description}
	\item[Masking] Microfacet is not visible from the viewer
	\item[Shadowing] Microfacet is not reachable from the light source
	\item[Interrecflection] Bounces between the microfacets
\end{description}

Many variations to the Cook-Torrance model have been developed, such as its predecessor Torrance-Sparrow~\cite{Torrance1967TheoryFO} or Oren-Nayar~\cite{oren1994generalization} for diffuse reflectance.

In this thesis, we focus on the distribution functions of the microfacets as they are ultimately the deciding factor of the rough surface look. A nice comparison of the three commonly used microfacet distributions --- \emph{Phong}, \emph{Beckmann} and \emph{GGX} -- along with their distribution functions, masking functions and sampling equations can be found in the \citet{walter2007microfacet}. As the exact formulations of those three methods are not a necessity for this thesis, we provide only a brief overview for each of them and an illustrative comparison between the GGX and the Beckmann of the same roughness in 
\autoref{fig:ggx_beckmann}.

\paragraph{Phong}

Even though the Phong distribution is purely empirical (not physically based), it is still quite popular choice for the microfacet distribution as it is simple to implement and provides sufficient results.

\paragraph{Beckmann}

The Beckmann distribution~\cite{beckmann1987scattering} is already physically based and for a long time has been considered the best solution to the rough surfaces as it is based on the Gaussian roughness. However, with the parameters set appropriately, it still provides the results very similar to the Phong distribution.

\paragraph{GGX}

The GGX distribution~\cite{walter2007microfacet} was introduced as an improvement over the Beckmann's solution for some cases. It maintains stronger tails, thus better shadowing and is based on the measured data of the real rough materials.

\begin{figure}[h]
	\centering
	\includegraphics[width=.8\linewidth]{img/ggx_beckmann.png}
	\caption{A rough aluminium sphere with the GGX distribution (left) compared to its Beckmann equivalent (right) rendered  in Mitsuba2}
	\label{fig:ggx_beckmann}
\end{figure}

\section{Polarization}
\label{sec:polarization}

Similarly to all kinds of the electromagnetic radiation, the light also propagates through space as a wave. The oscillation direction of this wave neither defines nor modifies the color of the light but it makes the light behave differently upon an interaction with certain materials. \autoref{fig:oscillation} explains the different directions of a wave.

\begin{figure}[h]
	\centering
	\includegraphics[width=.6\linewidth]{img/oscillation.pdf}
	\caption{Demonstration of the direction of oscillation and the direction propagation}
	\label{fig:oscillation}
\end{figure}

In the light's natural state (sun, common light bulb), the directions of its oscillation are arbitrary --- such light is called \emph{unpolarized}. The \emph{polarized} light maintains a restricted direction of oscillation and it is a result of a \emph{polarization} process. Note that the light is often only partially polarized as the restriction of the direction does not have to be perfect and allows some variations.

Depending on the shape of their electric fields, we distinguish three types of polarization which are demonstrated in \autoref{fig:polar_types}.

\begin{figure}[h]
	\centering
	\includegraphics[width=.7\linewidth]{img/polar_types.png}
	\caption[polar types]{An illustrative demonstration of the different types of polarization \footnotemark}
	\label{fig:polar_types}
\end{figure}
\footnotetext{\url{http://hyperphysics.phy-astr.gsu.edu/hbase/phyopt/polclas.html}}

To create a linearly polarized light, one may put a dielectric object in the direction of propagation of an unpolarized light. Due to the optical properties of the dielectrics, the reflected and the transmitted light will be polarized proportionally to the angle of the incidence. The angle at which the reflected light is perfectly polarized is called the \emph{Brewster's angle}~\cite{brewster1815laws}. It is computed by the following formula:

\begin{equation}
\theta=arctan(\frac{n_2}{n_1})
\end{equation}
, where $n_1$ is the IOR of the incident medium and $n_2$ of the transmitted medium.

In this case, we distinguish two types of the linearly polarized light depending on their relative orientation to the incident plane. The reflected light is called \emph{p-polarized} as its oscillation is parallel to the plane of incidence. The transmitted light is called \emph{s-polarized} as its oscillation is perpendicular (from the German word) to the plane of incidence. Both are shown in \autoref{fig:brewster}.

\begin{figure}[h]
	\centering
	\includegraphics[width=.7\linewidth]{img/brewster.pdf}
	\caption{Prefectly p-polarized reflected and partially s-polarized light refracted from a dielectric interface at the Brewster's angle}
	\label{fig:brewster}
\end{figure}


The principle of Brewster's angle is used in a material called the \emph{polarizer}. As the name suggests, it polarizes the light, restricting its direction of oscillation accordingly to the specifics of the polarizer. If the light that passes through the polarizer is of the opposite (perpendicular) direction to the polarizer's transmission orientation, it won't let it through and no light is visible. \autoref{fig:polarizer} illustrates the effects of a polarizer.

\begin{figure}[h]
	\centering
	\includegraphics[width=.6\linewidth]{img/polarizer.png}
	\caption[nikon]{Unpolarized light passes through a vertical polarizer $\rightarrow$ linearly vertically polarized light passes through a horizontal polarizer $\rightarrow$ no light\footnotemark}
	\label{fig:polarizer}
\end{figure}
\footnotetext{\url{https://www.apioptics.com/about-api/resources/visible-light-linear-polarizer/}}

This property is frequently incorporated in the sunglasses or camera filters to reduce the glare of the sun reflected from a horizontal surface. The reflected p-polarized light goes through a polarizer with a perpendicularly oriented transmission axis which consequently eliminates the incoming light. The effect of a polarizing filter is shown in \autoref{fig:polarizer_nikon}.

\begin{figure}[h]
	\centering
	\includegraphics[width=.7\linewidth]{img/polarizer_nikon.jpg}
	\caption[nikon]{Polarizing filter by Nikon\footnotemark}
	\label{fig:polarizer_nikon}
\end{figure}
\footnotetext{\url{https://www.nikonusa.com/en/learn-and-explore/a/tips-and-techniques/polarizing-filters-add-pow-to-pictures.html}}

As the polarization is a vast topic, we do not need to go into further detail for the purposes of this thesis. If the reader wishes to learn more, please refer to a scientific literature, for example \citet{kliger2012polarized}

\subsection{Polarization in rendering}
The integration of the polarization in the rendering process is quite rare as only a few scenarios display the effects of the polarization and one must implement quite complex behavior of the radiation waves to the spectral rendering. However, renderers such as Mitsuba2 or ART fully track the polarization state of light when needed. The implementation covered by this section is already in Mitsuba2~\cite{mitsubaWeb}.

The polarization state is represented by \emph{Stokes vector} --- a 4-dimensional quantity that fully parameterizes the elliptical shape of the light's electric field for each wavelength separately. The information stored in the Stokes vectors is explained in \autoref{fig:stokes}.

\renewcommand\thesubfigure{\arabic{subfigure}}
\begin{figure}
	\centering
	\begin{tabular}{cc}
		\begin{subfigure}
			{0.4\textwidth}\centering\includegraphics[width=\linewidth]{img/stokes1.png}
			\caption{Radiance - no polarization}
		\end{subfigure}
		&
		\begin{subfigure}
			{0.4\textwidth}\centering\includegraphics[width=\linewidth]{img/stokes2.png}
			\caption{Horizontal vs. vertical polarization}
		\end{subfigure} \\
		\begin{subfigure}
			{0.4\textwidth}\centering\includegraphics[width=\linewidth]{img/stokes3.png}
			\caption{Diagonal polarization}
		\end{subfigure} 
		&
		\begin{subfigure}
			{0.4\textwidth}\centering\includegraphics[width=\linewidth]{img/stokes4.png}
			\caption{Left vs. right circular polarization}
		\end{subfigure}
	\end{tabular}
	\caption{Different information carried by the Stokes vector}
	\label{fig:stokes}
\end{figure}

As we have the polarization states properly represented and we can track them throughout the rendering process, the next step is to determine the affects to these states upon a surface interaction. A transformation between the incoming state and the outgoing state is represented by the \emph{Mueller matrix} $M \in \R^{4x4}$. Due to the adjustments to all interactions in the light transport, we also generalize the BSDF $f_r(\lambda,\omega_i,\omega_o)$ to the polarized pBSDF $M(\lambda,\omega_i,\omega_o)$.

If the  reader is curious about the complications this implementation brings and their solutions, he may want to look into the details of Mitsuba2 in \citet{nimier2019mitsuba}. Nevertheless, they are not crucial for the purposes of this thesis and we purposely skip them.

\section{Dispersion}

Generally, the IOR of a dielectric at least slightly varies for different wavelengths (e.g. glass has IOR between 1.5 and 1.6). This causes that the polychromatic light is split into its spectral components upon an intersection with such materials. As each wavelength is slightly shifted, a rainbow effect can be perceived --- this phenomenon is called the \emph{dispersion}. In the nature, it can be frequently seen when light passes through liquids (sun through rain). For the scientific purposes, several variations of a device called the dispersive prism~\ref{fig:dispersion} have been created for this purpose. 

\begin{figure}[h]
	\centering
	\includegraphics[width=.6\linewidth]{img/dispersion.jpg}
	\caption[nikon]{Photograph of a dispersive prism\footnotemark}
	\label{fig:dispersion}
\end{figure}
\footnotetext{\url{https://en.wikipedia.org/wiki/Dispersive_prism}}

In the computer graphics, even though it is possible to simulate the dispersion in the tristimulus rendering, it is insufficient and the obvious choice would be to use the spectral rendering as it already contains most of the information about the tracking of the wavelengths. 

The missing part is the representation of the varying IOR --- currently, the widely used method is called \emph{Sellmeier approximation}~\cite{wilkie2002tone}. Then, we have to add the ability to track the possibly dispersed monochromatic rays upon a surface interaction of a single polychromatic ray, i.e. create extra samples that were unnecessary before.

\section{Iridescence}
\label{sec:irid}

It is quite common that some objects in the nature exhibit an interesting behavior where the hue of their surface graudally changes with the viewing angle and the illumination angle, such as butterfly wings, soap bubbles, oil etc. This phenomenon is called \emph{iridescence} or \emph{goniochromism} and is caused by a large amount of interferences between the light waves and their consequent scattering depending on their wavelength which produces a rapid change in colors~\cite{belcour2017practical}.

We distinguish two main types of the iridescence:

\begin{description}
	\item[Microscopic structures] Reflections from structures of a size similar to the wavelength (e.g. peacock feathers)
	\item[Thin film] Light interaction with a thin film of a size similar to the wavelength (e.g. soap bubble)
\end{description}

An example of both can be seen in \autoref{fig:iridescent_example}.

\begin{figure}
	\centering
	\begin{tabular}{cc}
		\includegraphics[width=0.4\linewidth]{img/iridescent_peacock.jpg}
		&
		\includegraphics[width=0.4\linewidth]{img/iridescent_soap.jpg}
	\end{tabular}
	\caption[Irid example]{Structural iridescence of the peacock feathers (left) and the thin-film light interference in a soap bubble (right)\footnotemark}
	\label{fig:iridescent_example}
\end{figure}
\footnotetext{\url{https://en.wikipedia.org/wiki/Iridescence}}

In this thesis, we focus on the thin-film interference as it is nicely described as a physical process and it is already incorporated in Mitsuba~\cite{belcour2017practical}. From now on, by iridescence we mean the thin-film interference until told otherwise.

First, look at the light interactions inside the membrane of a soap bubble in \autoref{fig:soap}. The light strikes at the surface of the film, based on the angle it can be either reflected or transmitted. The transmitted light very quickly strikes the bottom boundary of the soap bubble (as it is very thin) and again can be reflected and/or refracted. As the film is a few hundreds of nanometers thick, this repeats with a great frequency and, as you can see, the light transmitted from the upper boundary can easily interfere with the light reflected from the lower boundary.

\begin{figure}[h]
	\centering
	\includegraphics[width=.6\linewidth]{img/soap.pdf}
	\caption{A cross-section of a light interaction with a soap bubble}
	\label{fig:soap}
\end{figure}

An obvious observation is that the iridescence is also dependent on the thickness of the interacting layer - as the thickness increases, the transmission of the light takes longer time which consequently causes a lot less interferences. The difference between two variously thick films is displayed in \autoref{fig:irid_heights}.

\begin{figure}[h]
	\centering
	\includegraphics[width=.6\linewidth]{img/irid_heights.png}
	\caption{Two identical rough conductors with differently thick film layers on top of them: 550nm (left) vs. 1500nm (right) rendered in Mitsuba2}
	\label{fig:irid_heights}
\end{figure}

\subsection{Iridescence in rendering}

Based on the \citet{belcour2017practical}, we overview the computational process of the iridescence caused by a thin film on top of a rough material. We purposely avoid the exact formulations of the equations as these would be unnecessarily complicated to explain and it is sufficient to comprehend the basics in order to evaluate the correctness of the computation. For more details, the interested reader is referred to the \citet{belcour2017practical}.

Essentially, this procedure computes an iridescence term of the thin film layer that is plugged into the BSDF of the rough base surface:

\begin{enumerate}
	\item Compute the reflected and the transmitted values of the Fresnel equations for the IOR of the film and the IOR of the exterior
	\item Compute the optical paths differences between the primary and the secondary light paths
	\item Evaluate the Fresnel phase shift
	\item Determine the term by using the Airy summation for the parallel and perpendicular polarization
\end{enumerate}

\section{Fluorescence}

Curiously, certain materials or substances change their colors with no apparent respect to the illumination color. This behavior is called \emph{fluorescence} and it can be quite commonly observed in the nature, for example in various minerals but also in the living organisms such as fish or arachnids.

The explanation behind this phenomenon is that the molecules of such substances absorb electromagnetic radiation of specific wavelengths and emit back different, usually larger wavelengths. The most eye-catching fluorescence is caused by the absorption of the ultraviolet light which is invisible to the human eye and therefore the fluorescent material might seem to change its color for no reason. An example of a fluorescent calcite is shown in \autoref{fig:calcite}

\begin{figure}
	\centering
	\includegraphics[width=0.8\linewidth]{img/calcite.png}
	\caption[calcite]{Calcite under different lights\footnotemark}
	\label{fig:calcite}
\end{figure}
\footnotetext{\url{https://www.naturesrainbows.com/single-post/2017/11/01/Fluorescent-Multi-Wave-Calcite-from-the-Elmwood-Mine}}

Please note that there is a difference between fluorescence, luminescence and phosphorescence:

\begin{description}
	\item[Luminescence] Natural production of light caused by chemical reactions (no absorption)
	\item[Phosphorescence] Similar to fluorescence but emits light even after the light source is gone
	\item[Fluorescence] The emission stops almost instantaneously after the light source is gone
\end{description}

\subsection{Fluorescence in rendering}
As we are dealing with the wavelengths of a spectrum, the appropriate decision is to extend the spectral rendering to include the fluorescence. Once again, we refer the interested reader to \citet{mojzik2018handling} for the implementation details as we are only covering the fundamental ideas for the purposes of this thesis.

As we've mentioned before, we are in fact shifting the wavelengths of the absorbed spectrum to emit a new one --- this is called \emph{Stokes shift} (shown in \autoref{fig:stokes_shift}) and can be described by \emph{fluorescence response} $\Phi(\lambda_i,\lambda_o)$.

\begin{figure}
	\centering
	\includegraphics[width=0.7\linewidth]{img/stokes_shift.pdf}
	\caption{Illustration of the Stokes shift}
	\label{fig:stokes_shift}
\end{figure}

The discrete form of a fluorescence response can be represented by \emph{re-radiation matrix} which contains incoming wavelengths on its vertical axis and their corresponding outgoing wavelengths on its horizontal axis. The probability of the shift follows Kasha's rule --- the spectrum distribution of the emitted light should not change, only the intensity of emission spectrum does.

A generalization of the BRDF that includes re-radiation was introduced by \citet{hullin2010acquisition} called \emph{Bi-spectral Bidirectional Reflectance and Re-radiation distribution function (bi-spectral BRRDF)}:
\begin{equation}
f_r((\omega_i,\lambda_i),(\omega_o,\lambda_o))=\frac{d^2L(\omega_o,\lambda_o)}{L(\omega_i,\lambda_i)d\omega_i d\lambda_i}
\end{equation}
and the corresponding \emph{bi-spectral rendering equation}:
\begin{equation}
L(\omega_o,\lambda_o)=\int_{\Lambda}\int_{\Omega}L(\omega_i,\lambda_i)f_r((\omega_i,\lambda_i),(\omega_o,\lambda_o))d\omega_i d\lambda_i
\end{equation}

The topics of this chapter properly explained the appearance phenomena that we are investigating in this thesis. As we covered the basics of their computations as well, we may now proceed with the evaluation process to determine their accuracy.