\chapter*{Conclusion}
\addcontentsline{toc}{chapter}{Conclusion}

Both ART and Mitsuba2 display several unique properties and implementation details such us custom spectral sampling techniques. Despite that, we've shown that it is possible to methodically test the appearance computations of the distinct renderers. As the models behind each of the tested sensation is properly defined and we can describe their natural behavior, it is possible to expose some of the exact aspects of the computations (e.g Brewster's angle in polarization) and demonstrate their functionality.

Even though we do not provide an absolute metric that would confirm the correctness with a simple yes/no answer, the reference images along with the difference images provide enough information for a reasonably skilled user to asses the accuracy ratio by himself.

Even though the benchmark is usable to test the specific phenomena, we are aware of its shortcomings and would like to mention them in future work. 

\section{Future Work}

There are several possible extensions that we consider interesting or useful but that were not essential for the purposes of this thesis so we purposely avoided them. Providing more time, these would be a fine asset to the benchmark, further extending its capabilities and effectiveness.

\begin{description}
	\item[Enhanced results] Right now, the results visualizer consists of a very basic UI where the user may look at the images and interact with them. We would like to add several features, e.g. performance counter, comments explaining each scene, highlights of the scene, etc.
	\item[Dispersion] As the dispersion is the only phenomenon that we've talked about but haven't evaluated, it would be appropriate to add it to the benchmark as soon as the implementation for Mitsuba2 and/or ART is fully functional.
	\item[More renderers] The addition of multiple renderers heavily depends on the supported features of the specific renderer and on the interest of it's developers. However, if we find a renderer that supports at least a majority of the features that we evaluate, we would gladly include it in the renderer.
	\item[Common scene format] Including more renderers would be significantly simplified by describing the scenes in a common scene format (e.g. Universal Scene Description by~\citet{usdDoc}). This approach would, of course, need a conversion tool from the universal format to the renderer-specific one.
\end{description}